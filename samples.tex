

\documentclass{article}
\usepackage[utf8]{inputenc}
\usepackage[T2A]{fontenc}
\usepackage[russian]{babel}
\usepackage{graphicx}
\usepackage{amsfonts}
\usepackage{amsmath}
\usepackage{amssymb}

\title{ТФКП, M3238-39}
%\author{murzinanastasiia}
\date{\today}

% auto numeration as <section>.<task>
\renewcommand*{\theenumi}{\thesection.\arabic{enumi}}
\renewcommand{\labelenumi}{\theenumi}

% additional macroses
\providecommand{\abs}[1]{\left\lvert#1\right\rvert} 
\newcommand{\theIm}{\operatorname{Im}}
\newcommand{\conj}[1]{\overline{#1}}

\begin{document}

\maketitle

\section*{Как находить $\frac{1}{z}$?}

В случае прямых (или отрезков): подставить уравнение прямой в $1/z$, выделить вещественную и мнимую часть. Для окружностей использовать параметризацию $Az\overline{z}+Bz+C\overline{z}+D=0$. Из чисто дробной части функции $w(z)$ выразим $z$ и $\overline{z}$ - подставим их в уравнение окружности.

\section*{Как устранить разрез по отрезку в верхней полуплоскости?}

Отобразить верхнюю полуплоскость с разрезом по отрезку $[a, a + ih]$ можно так: $w = \sqrt{(z-a)^2+h^2}$.

\section*{Как использовать обратное преобразования для нахождения функций при устранении разреза?}

Область $Imz>0, \vert z \vert >R$ на верхнюю полуплоскость переводится функцией $w=\frac{1}{2}(\frac{z}{R}+\frac{R}{z})$.

\section*{Как устранять разрез внутри окружности? Обратите внимание на такие разрезы при подготовке к переписыванию!}

Круг $\vert z \vert < r$ с разрезом по отрезку $[0, r]$ на круг $\vert w \vert < 1$ можно перевести функцией: $w=\frac{z + r + i 2\sqrt{rz}}{z + r - i 2\sqrt{rz}}$.

\section*{Что о тригонометрических функциях?}

Полуполосу $Im z > 0$,$ 0 < Re z < h$ на верхнюю полуплоскость переводит функция $w=-\cos \dfrac{\pi z}{h}$ (тригонометрические функции -- суперпозии экспонент и линейных функций, конечно, можно использовать несколько преобразований, это лишь пример как действуют тригонометрические функции).

\section*{Что обычно идёт не так?}

Часто в устранении разрезов хочется использовать не конформные функции. Следует помнить, что у некоторых функций ($z^n$, $e^z$...) есть области конформности.


\end{document}
