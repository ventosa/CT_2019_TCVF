

\documentclass{article}
\usepackage[utf8]{inputenc}
\usepackage[russian]{babel}

\title{ТФКП, M3238-39}
%\author{murzinanastasiia}
\date{upd 14 февраля 2019}

\usepackage{natbib}
\usepackage{graphicx}
\usepackage{amsfonts}

\begin{document}

\maketitle

\section{Комлексные числа}
1.1 Решить уравнение $\bar{z} = z^{n-1}, (n \neq 2)$\\ \\
1.2 Доказать, что оба значения $\sqrt{z^2-1}$ лежат на прямой, проходящей через начало координат и параллельной биссектрисе внутреннего угла треугольника с вершинами в точках $-1, 1$ и $z$, проведённой из вершины $z$.\\ \\
1.3 Доказать, что $(^n\sqrt{z})^m$ ($n, m$ - целые числа, а $(n,m)$ - наибольший общий делитель) имеет $\frac{n}{(n, m)}$ различных значений \\ \\
1.4 Доказать $\vert 1 - \bar{z_1} z_2 \vert^2 - \vert z_1 - z_2 \vert ^2 = (1 - \vert z_1 \vert ^2) (1 - \vert z_2 \vert ^2)  $\\ \\
1.5 Доказать, что если $\vert z_1 + z_2 + z_3 \vert  = 0 $  и $\vert z_1 \vert = \vert z_2 \vert = \vert z_3 \vert= 1$, то точки $z_1, z_2, z_3$ являются вершинами правильного треугольника \\ \\
1.6 Изобразить область или прямую: \begin{itemize}
    \item $\vert z-2 \vert^2 - \vert z+2 \vert^2 > 3$;
    \item $log_{\frac{1}{2}}\frac{\vert z - 1 \vert + 4}{3\vert z - 1\vert -2} > 1$;
    \item ${Im}(\overline{z^2-z})=2-{Im} z$;
    \item $\vert z \vert - 3{Im} z=6$;
    \end{itemize}
    \\
1.7 Как действует отображение $e^z$ на прямую $x=y$, прямую $(y=const, x\in R)$, полосу $y\in(\phi, \psi), x\in R$ ?\\
1.8 Какая функция отображает полуплоскость $Im z > 0$ в окружность единичного радиуса с центром в начале координат, причём $z_0 \rightarrow (0; 0)$?

\end{itemize} 




\end{document}
