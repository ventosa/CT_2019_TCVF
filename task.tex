

\documentclass{article}
\usepackage[utf8]{inputenc}
\usepackage[T2A]{fontenc}
\usepackage[russian]{babel}
\usepackage{graphicx}
\usepackage{amsfonts}
\usepackage{amsmath}
\usepackage{amssymb}

\title{ТФКП, M3238-39}
%\author{murzinanastasiia}
\date{\today}

% auto numeration as <section>.<task>
\renewcommand*{\theenumi}{\thesection.\arabic{enumi}}
\renewcommand{\labelenumi}{\theenumi}

% additional macroses
\providecommand{\abs}[1]{\left\lvert#1\right\rvert} 
\newcommand{\theIm}{\operatorname{Im}}
\newcommand{\conj}[1]{\overline{#1}}

\begin{document}

\maketitle

\section{Комлексные числа}
1.1 Решить уравнение $\bar{z} = z^{n-1}, (n \neq 2)$\\ \\
1.2 Доказать, что оба значения $\sqrt{z^2-1}$ лежат на прямой, проходящей через начало координат и параллельной биссектрисе внутреннего угла треугольника с вершинами в точках $-1, 1$ и $z$, проведённой из вершины $z$.\\ \\
1.3 Доказать, что $(^n\sqrt{z})^m$ ($n, m$ - целые числа, а $(n,m)$ - наибольший общий делитель) имеет $\frac{n}{(n, m)}$ различных значений \\ \\
1.4 Доказать $\vert 1 - \bar{z_1} z_2 \vert^2 - \vert z_1 - z_2 \vert ^2 = (1 - \vert z_1 \vert ^2) (1 - \vert z_2 \vert ^2)  $\\ \\
1.5 Доказать, что если $\vert z_1 + z_2 + z_3 \vert  = 0 $  и $\vert z_1 \vert = \vert z_2 \vert = \vert z_3 \vert= 1$, то точки $z_1, z_2, z_3$ являются вершинами правильного треугольника \\ \\
1.6 Изобразить область или прямую: \begin{itemize}
    \item $\vert z-2 \vert^2 - \vert z+2 \vert^2 > 3$;
    \item $log_{\frac{1}{2}}\frac{\vert z - 1 \vert + 4}{3\vert z - 1\vert -2} > 1$;
    \item ${Im}(\overline{z^2-z})=2-{Im} z$;
    \item $\vert z \vert - 3{Im} z=6$;
    \end{itemize}
    \\
1.7 Определить семейство линий в $z$ -плоскости($-\infty < C < \infty$), заданных уравнениями:
\begin{itemize}
	\item $Re \dfrac{1}{z}=C$
	\item $Im \dfrac{1}{z}=C$
	\item $\dfrac{\vert z - z_1 \vert}{\vert z - z_2 \vert}=\lambda, (\lambda > 0)$
\end{itemize}
    
1.8 Доказать, что многочлен $f(x) = (\cos \alpha + x \sin \alpha)^n - \cos n\alpha - x\sin n\alpha$ делится на $x^2+1$.

1.9 Найти наибольшее и наименьшее расстояния от начала координат до линии $\vert z + \dfrac{1}{z}\vert = a, (a>0)$

1.10 Первоначальное значение $Arg f(z)$ при $z=2$ принято равным 0. Точка $z$ делает один оборот против часовой стрелки по окружности с центром в начале координат и возвращается в точку $z=2$. Считая, что  $Arg f(z)$ меняется непрерывно при движении точки $z$, указать значение $Arg f(2)$ после указанного поворота, если: \begin{itemize}
	\item $f(z) = \sqrt{z-1}$
	\item $f(z) = \sqrt{z^2+2z-3}$
	\item $f(z) = \sqrt{\dfrac{z-1}{z+1}}$
\end{itemize} \\

1.11 Доказать, что $\frac{x^{2m}-a^{2m}}{x^2-a^2}=\prod_{k=1}^{m-1}(x^2-2ax\cos \frac{k\pi}{m}+a^2)$. \\

1.12 Найти на сфере Римана образы окружностей с центром в начале координат \\

1.13 Найти на комплексной плоскости образ параллели с широтой $\phi, (\pi/2\leq\phi\leq\pi/2)$. \\

1.14 Найти суммы $S_n = 1 + \frac{\sin x}{\sin x}+\frac{\sin 2x}{\sin^2 x}+...+\frac{\sin nx}{\sin^nx},
\sigma_n = 1 + \frac{\cos x}{\sin x}+\frac{\os 2x}{\sin^2 x}+...+\frac{\cos nx}{\sin^nx}$

1.15 Решить систему уравнений \begin{cases} z^3+w^5=0 \\ z^2\bar{w}^4=1  \end{cases}

1.16 Найти все корни следующих уравнений:
\begin{eqnarray}
\sin z + \cos z = 2;\\
\sin z - \cos z = 3;\\
\sh z - \ch z = 2i;\\
2\ch z + \sh z = i;\\
\cos z = \ch z;\\
\cos z = i \sh 2z.
\end{eqnarray}

\section{Отображения}
2.1 Как действует отображение $e^z$ на прямую $x=y$, прямую $(y=const, x\in R)$, полосу $y\in(\phi, \psi), x\in R$ ?\\
2.2 Какая функция отображает полуплоскость $Im z > 0$ в окружность единичного радиуса с центром в начале координат, причём $z_0 \rightarrow (0; 0)$? \\
2.3 Найти образы координатных осей ОХ и ОУ при преобразовании $w= \dfrac{z+1}{z-1}$.\\
2.4 Найти линейное преобразование, оторбражающий треугольник с вершинами $0, 1, i$  на подобный ему с вершинами $0, 2, 1+i$.\\
2.5 Найти линейную функцию, отображающую круг $\vert z \vert < 1$  на круг $\vert w - w_0 \vert < R$ так, чтобы центры кругов соответсвовали друг другу и горизонтальный диаметр переходил в диаметр, образующий угол $\alpha$ с направлением дейтвительной оси.\\
2.6 Построить область на плоскости $w$, в которую отображается угол $0<\phi<\pi/4$ с помощью функции $w=\frac{z}{z-1}$.\\
2.7 Во что преобразуется окружность $\vert z \vert =1 $ при отображении $w=\frac{1-z}{z}$?\\
2.8 В какую область преобразуется круг $\vert z - 1/2 +i/2\vert\leq\frac{\sqrt{2}}{2}$ при отображении $w=\frac{iz-2}{z+i}$?\\
2.9 Найти в какую область преобразуется квадрат $0\leq x \leq1, 0\leq y \leq1$ функцией $w=z^2+z+1$.\\
2.10 Найти образ плоскости с разрезом вдоль положительной части вещественной оси при отображении однозначной ветвью логарифма, когда $z_0=i$ переходит в $w_0=\frac{5}{2}\pi i$.\\
2.11 Отобразить треугольник, заключённый между прямыми $y=2, x=0, y=x$ с помощью функции $w=1/2z^2-z$ на плоскость $w$.\\
2.12 Найти однолистное и конформное отображение вертикальной полосы $1 < Re z < 2$  на верхнюю полуплоскость $Im w > 0$.\\


\section{Разрезы} 
3.1 Найти однолистное и конформное отображение полосы $0 < Im z < \pi$ с разрезом $-\inf < Re z \leq 0, {Im} z =\frac{\pi}{2}$ на полосу $0 < {Im} w < \pi$.\\
3.2 Найти функцию $w(z)$, конформно отображающую всю плоскость $z$ c разрезом по дуге окружности $\vert z \vert = 1, Im {z}>0$, на всю плоскость $w$ с разрезом по отрезку $[-1, 1]$ и удолетворяющую условиям $w(1)=1, w(\infty)=\infty$.\\
3.3 Найти функцию $w(z)$, конформно отображающую полукруг $\vert z \vert < 1, Im {z}>0$, на всю плоскость $w$ на круг $\vert w \vert < 1$. \\
3.4 Найти функцию $w(z)$, конформно отображающую разрезанную по отрезку $[0, i]$ полуплоскость $Im {z}>0$  на круг $\vert w \vert < 1$ и удолетворяющую условиям $w(\frac{5i}{4})=0, w(i)=-i$.\\
3.5 Найти однолистное и конформное отображение верхней полуплоскости с разрезом по отрезку от точки $z_1 = 0$ до точки $z_2 = i$ на верхнюю полуплоскость $Im w > 0$.\\
3.6 Найти функцию $w(z)$, конформно отображающую круг $\vert z \vert < 1$, разрезаннй по радиусу $[1/3, 1]$ на круг $\vert w \vert < 1$.\\
3.7 Найти функцию $w(z)$, конформно отображающую верхнюю полуплоскость $Im z > 0$ с вырезанной точкой $z=ih$, где $h$ - вещественное число, на верхнюю полуплоскость $Im z > 0$.\\
3.8 Найти функцию, отображающую область $\frac{\pi}{4} < arg(z) < \frac{3\pi}{4}$ с разрезом по мнимой оси в виде луча $Im z>h, Re z = 0$ на область $Im z>0$.\\ 
3.9 Найти какую-либо функцию $w(z)$, конформно отображающую область ${Im(z)>0, z\not\in[k\pi, k\pi+i\pi], (k=0, \pm 1, ...)}$, на верхнюю полуплоскость.\\
3.10 Найти конформное отображение расширенной плоскости $z$ с разрезом вдоль дуги АВ окружности, концы которой лежат в точках $\pm a$ вещественной оси (внешность дуги АВ), на внешность круга расширенной плоскости $W$, граница которого проходит через те же точки $\pm a$ (\emph{построение профилей Жуковского})\\

\section{Пределы функции}
4.1 Найти все точки, в которых дифференцируемы функции:\begin{itemize}
\item $Re z$;
\item $x^2+iy^2$;
\item $tg z$;
\item $\frac{cos z}{cos z - sin z}$;
\end{itemize}\\
4.2 Вывести условия Коши-Римана для предстаавления комплексных чисел в  полярных координатах. \\
4.3 Функция $w = \frac{Im z}{\overline{z}}$ определена для $z \neq 0$. Можно ли доопределить ее в точке $z = 0$ так, чтобы она стала непрерывной в этой точке? % рассмотреть предел вдоль приямой y=kx \\
4.4 Вычислить пределы последовательностей
\begin{itemize}
\item $z_n = (\frac{n-2}{n+2})^{-3n} + i(1+\frac{5}{n-2})^{4n};$
\item $z_n=\frac{sin{5z}}{z^2+sh{3z}};$
%\item $;$
\end{itemize}
\\
4.5 Найти области дифференцируемости функций:
\begin{itemize}
\item $w=z+i\overline{z};$
\item $Re(w) = arctg \frac{y}{x} + x^2 + y^2;$
\item $w=\frac{z+5i}{iz-7};$
\item $w=e^\overline{z}^2;$
\item $w=\vert x^2 - y^2 \vert +2i\vert zy \vert;$
\item $w=sin{x}sh{y} -i cos{x}ch{y};$
\end{itemize}\\
4.6 Проверить аналитичность и восстановить, где это возможно функцию:
\begin{itemize}
\item $v = arctg \frac{y}{x} +e^xsin{y}+3y;$
\item $v=e^{3x}cos{3y};$
\item $u=ln(x^2+y^2)+x^2;$
\end{itemize}
4.7 Разложить в ряд Тейлора в окрестности точки $a$ и определить область сходимости:
\begin{itemize}
\item $z^3, a=2$;
\item $e^z, a=3$;
\item $\frac{1}{1-z^2}, a=0$;
\item $\frac{1}{z}, a=1$;
\item $\frac{z}{(1-z^2)^3}, a=0$;
\item $f(z)=\sqrt{z+3}, a=1,f(1)=-2$;
\item $f(z)=Arcsinz,a=0, f(0)=4\pi$.
\end{itemize}\\
4.8 Представить рядом Лорана по степеням $(z-a)$ и определить
область сходимости:
\begin{itemize}
\item $\frac{z}{(z-a)^n}, a=a$;
\item $\frac{1}{z^2(z-b)^3}, a=0, (b=const)$;
\item $z^3cos\frac{1}{z},a=0$;
\item $z^5e^\frac{1}{z},a=0$;
\item $e^zln\frac{z-\alpha}{z-\beta}, a=0, max(\vert\alpha\vert, \vert \beta \vert) < \vert z\vert < \infty$;
\item $$;
\item $$;
\end{itemize}
4.9 Разложить в ряд Фурье (рассмотреть как ряд Лорана с $\sum c_nz^n, z=e^{i\phi}$ на окружности $\vertz\vert=1$):
\begin{itemize}
\item $\frac{1-acos\phi}{1-2acos\phi+a^2}, -1<a<1$;
\item $\frac{1}{1-asin\phi}, -1<a<1$;
\item $\cos\phi\cdotln(1+a^2cos^2\phi), 0<a<1$;
\item $ln\vert sin \frac{\phi}{2} \vert$;
\item $\frac{\pi-\phi}{2}$.
\end{itemize}
Hint. $ln\vert sin \frac{\phi}{2} \vert = Re{ln\frac{1-e^{i\phi}}{2}}$
4.10 Найти множества точек, в которых сходятся следующие ряды Лорана:
\begin{itemize}
\item $\sum_{n=-\infty}^{\infty}2^{-\vertn\vert}z^n$;
\item $\sum_{n=-\infty}^{\infty}\dfrac{z^n}{3^n+1}$;
\item $\sum_{n=-\infty}^{\infty}\frac{(z-1)^n}{ch(\alphan)}, \alpha > 0$;
\item $\sum_{n=-\infty}^{\infty}\frac{z^n}{n^2+1}$;
\item $\sum_{n=-\infty}^{\infty}2^nz^n$;
\end{itemize}

4.11 Найти главную часть ряда Лорана в окрестности особой точки:
\begin{itemize}
\item $\frac{z}{(z+2)^2}$;
\item $\frac{e^z+1}{e^z-1}$;
\item $\frac{1}{sin(\piz)}$;
\item $\sqrt{z^4+b^4}, z_0=\infty$;
\item $z^3 arctgz, z_0=\infty$.
\end{itemize}

4.12 Доказать, что функция $f(z)$ регулярна во всей комплексной плоплоскости, за исключением конечного числа изолированных особых точек однозначного характера $z_1, z_2... z_n$ (точка $z = \infty$ не вклю-
включается). Доказать, что $$\sum_{k=1}^n res_{z=z_k}f(z)+res_{z=\infty}f(z)=0$$

\section{Интегрирование функции комплексного переменного. Вычеты}

5.1 Вычислить интеграл $\int (z-a)^ndz$, $n$-целое число, по
\begin{itemize}
\item полуокружности $\vert z - a \vert =R,0\leq arg(z-a)\leq \pi$, начало пути в точке $z=a+R$; 
\item по окружности $\vert z - a \vert =R$;
\item по периметру квадрата с центром в точке а и сторонам, паралелльным осям координат.
\end{itemize} 
5.2 Вычислить интегралы:
\begin{itemize}
\item $\int_{\vert z \vert = 1}z^nLnzdz$, где n-целое число, (а) $Ln(1)=0$, (б) $Ln(-1)=\pi i$;
\item $\int_0^{\inf} e^{-x^2}cos(2bx)dx$;
\item $\int_0^{\inf} \frac{sinx}{x}dx$;
\item $\int_0^{\inf} x^{s-1}cosxdx, 0<s<1, \gamma(t)=\int_0^{\inf} x^{t-1}e^{-x}dx$;
\end{itemize}
5.3 Доказать, что если $\vert a \vert \neq R$,то $$\int_{\vert z\vert = R} \frac{\vert dz \vert}{\vert z-a\vert \vert z+a \vert} < \frac{2\piR}{\vert R^2 - \vert a\vert^2\vert}$$.
5.4 Доказать, что "Для того чтобы изолированная особая точка $z_0$ функции $f(z)$ была существенно особой, необходимо и достаточно, чтобы главная часть лорановского разложения функции $f(z)$ в окрестности $z_0$ содержала бы бесконечное число членов: $f(z)=\sum_{n=-\infty}^{n=\infty}a_n(z-z_0)^n$".
\end{itemize}
5.5 Вычислить интегралы по следующим конутрам:
\begin{itemize}
\item $\int_{\partial D} \frac{dz}{1+z^4} (D: \vert z - 1 \vert < 1)$;
\item $\int_{\partial D} \frac{sin z}{(z+1)^3dz}(D: x^{2/3}+y^{2/3}<2^{2/3})$;
\item $\int_{\partial D} zsin\frac{z+1}{z-1}dz(D:\vert z \vert <2)$;
\item $\int_{\partial D} \frac{ctgzdz}{z}(D:\vert z \vert < 1)$;
\item $\int_{\partial D} \frac{e^{\piz}dz}{2z^2-i}(D:\vert z \vert < 1, Rez>0, Imz>0)$;
\item $\int_{\partial D} \frac{z^2dz}{e^{2\piiz^3}-1}(D:\vert z \vert < \sqrt[3]{n+\frac{1}{2}}(n=0,1,2...))$;
\end{itemize}



\end{document}
